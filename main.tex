%!TEX encoding = UTF-8 Unicode
%!TEX program = xelatex

\documentclass[doctor]{hnuthesis}
\usepackage{subfigure}
\usepackage{amsmath, unicode-math}
\DeclareMathAlphabet{\mathcal}{OMS}{cmsy}{m}{n}
\DeclareMathAlphabet{\mathbb}{U}{msb}{m}{n}%

%\newCJKfontfamily{\simsun}[AutoFakeBold = {3.17}]{SimSun} %%中文加粗

\usepackage{algpseudocode}
\usepackage{float}
\usepackage{makecell}

\usepackage{tikz}
\usepackage{etoolbox}
\newcommand{\circled}[2][]{\tikz[baseline=(char.base)]
    {\node[shape = circle, draw, inner sep = 1pt]
    (char) {\phantom{\ifblank{#1}{#2}{#1}}};%
    \node at (char.center) {\makebox[0pt][c]{#2}};}}
\robustify{\circled} 

\usepackage{listings}

%\usepackage{setspace}
%\linespread{1.5}



% 学校代码
\hnucode{10532}
% 学校名称
\hnuname{湖南大学}
\enhnuname{Hunan University}
% 中图分类号
\clc{TP391}         
% 密级       
\secrettext{普通}

% 标题
\title{针对国产ARM架构CPU的深度学习函数库的设计与优化}
\entitle{Design and optimization of deep learning function library for domestic ARM architecture CPU}
% 作者
\author{谭言西}
\enauthor{Yanxi Tan}
% 学号
\authorid{S2010W0916}      
% 学院
\college{信息科学与工程学院}
% 专业
\major{计算机技术} 
\enmajor{Computer Technology}
%学士学位获得学校,年份
\enbachelor{B.E.~(NanChang University)2018}
% \enmaster{M.S.~(Hunan University)2020}
\endoctor{Master of engineering}
% 研究方向
\workon{高性能计算、体系结构}
% 导师
\supervisor{全哲\ 教授}
\ensupervisor{Professor Zhe Quan}
% 论文提交、答辩日期
\submitdate{二〇二三年x月xx日}
\defensedate{二〇二三年x月xx日}
\endate{June, 2023}
% 答辩委员会主席
\chair{待定}



\usepackage[export]{adjustbox} \begin{document}
% 封面、原创性声明
\maketitle

% 摘要
%中文摘要
\begin{abstract} %TODO摘要没有居中
	中文摘要。文章
	\keywords{关键字1;关键字2;关键字n}
\end{abstract}

%英文摘要
\begin{enabstract}
	英文摘要。
	\enkeywords{Keyword1; Keyword2; Keywordn}
\end{enabstract}
% 目录
\tableofcontents
% 插图附表索引
\listoffigures
\listoftables

% 正文章节
\mainmatter
% \input{chapters-main/test.tex}
\chapter{绪\quad 论}

绪论。


\input{chapters-main/ch2}
\input{chapters-main/ch3}
\input{chapters-main/ch4}
\input{chapters-main/ch5}
% \input{chapters-main/samples}

\begin{summary}
	总结。引用 ResNet~\citep{2020_aradi_Survey}提出\citen{2020_aradi_Survey}。\citep{2021_chen_Graph, 2021_qiao_Reinforcement, 2021_di_survey, 2017_qinxiaohui_FeiYunZhiCheLiangDuiLieDeFenBuShiKongZhi}中文显示\citen{2017_qinxiaohui_FeiYunZhiCheLiangDuiLieDeFenBuShiKongZhi}\citen{2020_aradi_Survey}效果后\citet{2017_qinxiaohui_FeiYunZhiCheLiangDuiLieDeFenBuShiKongZhi}面的问下好的看看\citet{2021_chen_Graph}提出了\citen{2017_qinxiaohui_FeiYunZhiCheLiangDuiLieDeFenBuShiKongZhi}可行吗?
\end{summary}

\bibliography{references}

\appendix
\chapter{攻读学位期间所发表的学术论文}

\begin{enumerate}
    \item 第三类永动机
\end{enumerate}


\chapter{攻读学位期间所参加的科研项目}

\begin{enumerate}
    \item 第三类永动机
\end{enumerate}


\backmatter
\begin{acknowledgements}

	致谢。

\end{acknowledgements}


\end{document}

