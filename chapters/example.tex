\chapter{论文模板简单使用说明}
\section{Latex使用}
该模板已经在mac的texifier,overleaf,vscode上验证使用,请用XeLaTex进行编译。

\section{参考文献}

在Doctoral-thesis.tex修改参考文献链接:\lstinline!\bibliography{MyLibrary.bib}!。

目前可以使用的引用命令有\lstinline!\cite{}!、\lstinline!\citep{}!、\lstinline!\citet{}!和\lstinline!\citen{}!

效果如下:

引用 ResNet~\cite{2020_aradi_Survey}提出了ResNet网络结构\citep{2020_aradi_Survey}(上述第一个效果和第二个效果相同)。

以人名为主语:\citet{2017_qinxiaohui_FeiYunZhiCheLiangDuiLieDeFenBuShiKongZhi}针对队列稳定性提出了观测性方法。或者\citet{2021_chen_Graph}对与图做了个总结。

参考文献的引用图标不上标:\citen{2017_qinxiaohui_FeiYunZhiCheLiangDuiLieDeFenBuShiKongZhi}这样。

\section{图表}
插入图的常用命令有:
\begin{lstlisting}
\begin{figure}[htb]
  \centering
  \includegraphics[width=.7\textwidth, page=1]{figures/dc.pdf}
  \caption{决策控制框架}
  \label{fig:dc-1}
\end{figure}
\end{lstlisting}

引用的使用可以直接使用\lstinline!\figref{fig:dc-1}!

三线表的命令:
\begin{lstlisting}
\begin{table}[htb]
  \centering
  \caption{训练超参数}
  \label{tab:sarl-4}
  \begin{tabular}{ccc}
	\toprule
	物理意义 & 英文/符号 & 数值 \\ 
	\midrule
	 隐含层数 & Hidden layer number & $\rm 2$ \\
	\bottomrule
  \end{tabular}
\end{table}
\end{lstlisting}

引用格式为\lstinline!\tabref{tab:sarl-4}!

\section{定理类环境}
该模板定义了定理,引理,评注等,具体有:
\begin{lstlisting}
\newcommand\hnu@assertionname{断言}
\newcommand\hnu@axiomname{公理}
\newcommand\hnu@corollaryname{推论}
\newcommand\hnu@definitionname{定义}
\newcommand\hnu@propertyname{性质}
\newcommand\hnu@examplename{例}
\newcommand\hnu@lemmaname{引理}
\newcommand\hnu@proofname{证明}
\newcommand\hnu@propositionname{命题}
\newcommand\hnu@remarkname{评注}
\newcommand\hnu@theoremname{定理}
\newcommand\hnu@assumptionname{假设}
\end{lstlisting}

\section{公式引用}
引用格式为\lstinline!\eqref{}!

